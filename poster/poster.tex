%!TeX program = xelatex
\documentclass[final]{beamer}

% ====================
% Packages
% ====================

\usepackage[T1]{fontenc}
\usepackage{lmodern}
\usepackage{anyfontsize}
\usepackage[orientation=portrait,size=a0,scale=1.35]{beamerposter}
\usetheme{gemini}
\usecolortheme{unipd-light}

\usepackage{graphicx}
\usepackage{booktabs}
\usepackage{tikz}
\usepackage{tikzscale}
\usepackage{pgfplots}
\usepackage{steinmetz}
\usepackage{stmaryrd}
\usepackage{mathtools}
\usepackage{xpatch}
\usepackage{amsmath}
\usepackage{bbold}
\usepackage{multicol}
\usepackage{algorithm2e}
\usepackage[thicklines]{cancel}
\usepackage{amssymb}
\usepackage{color}
\usepackage{subfiles}

\xpatchcmd{\@algocf@start}% <cmd>
{-1.5em}% <search>
{0pt}% <replace>
{}{}% <success><failure>

\DeclareMathOperator{\sgn}{sgn}
\newcommand{\mathsc}[1]{\operatorname{#1}}




\xpatchcmd{\phase}{#2}{\vphantom{,}#2}{}{}
\tikzset{every picture/.style={line width=1.5pt}} %set default line width to 0.75pt
\pgfplotsset{compat=1.17}

% ====================
% Lengths
% ====================

% If you have N columns, choose \sepwidth and \colwidth such that
% (N+1)*\sepwidth + N*\colwidth = \paperwidth
\newlength{\sepwidth}
\newlength{\colwidth}
\setlength{\sepwidth}{0.0025\paperwidth}
\setlength{\colwidth}{0.49\paperwidth}

\newcommand{\separatorcolumn}{\begin{column}{\sepwidth}\end{column}}

% ====================
% Title
% ====================

\title{Parallel Computing}

\author{Simone}

\institute[shortinst]{UNIPD}

% ====================
% Logo (optional)
% ====================

% use this to include logos on the left and/or right side of the header:
\logoright{\includegraphics[height=7cm]{media/DEI}}
\logoleft{\includegraphics[height=7cm]{media/unipd_sigillo}}

% ====================
% Body
% ====================
\begin{document}
    \begin{frame}[t]
        \begin{columns}[t]
            \separatorcolumn
            \begin{column}{\colwidth}
                \begin{block}{Parallel Algorithms}
                    \heading{Parallel sorting}
                    \begin{description}
                        \item[Merge sort]  
                        \item[Bitonic Merging]  
                        \item[Bitonic Sorting] 
                    \end{description}

                    \heading{Prefix Computation}
                    \heading{Binary Adder} forse?
                    \heading{FFT for Powers of Two}
                    \heading{Benes Permutation}

                \end{block}
                \begin{block}{Network Definitions}
                    \heading{Degree}
                        In an undirected graph \(G = (V, E)\), the \textbf{degree of a node} \(a \in V\) is the number of its immediate neighbours, that is:
                        \begin{equation*}
                            \operatorname{degree}(a) = \left| \left\{ b \in V : (a,b) \in E \right\} \right|
                        \end{equation*}

                        The \textbf{degree of a graph} is the maximum degree of any of its nodes, that is:
                        \begin{equation*}
                            \operatorname{degree}(G) = \max_{a \in V} (degree(a))
                        \end{equation*}

                        A graph is said to be \textbf{regular of degree \(\Delta\) } if all its nodes have degree \(\Delta\)

                    \heading{Diameter}
                    \heading{Dichotomy}

                \end{block}
            \end{column}
            \separatorcolumn
            \begin{column}{\colwidth}
                \begin{block}{Linear Array and Ring}
                    \heading{Linear Array}
                    \heading{Ring}
                \end{block}
                \begin{block}{Binary Hypercube}
                    
                \end{block}
                \begin{block}{Binary Hypercube}
                    
                \end{block}
                \begin{block}{Multidimensional Meshes and Tori}
                    
                \end{block}
                \begin{block}{Embeddings and simulations}
                    
                \end{block}
                \begin{block}{Altri blocchi di embedding specifiche}
                    Tipo LA e H, H e M, M e LA presi anche da esami e soluzioni del prof oltre che dalle slide viste a lezione
                \end{block}
            \end{column}
        \end{columns}
    \end{frame}
\end{document}