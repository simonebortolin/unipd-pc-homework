\section{Matrix parallelization}\label{matrix_parallelization}

Although, the parallelisation of the «for \(i\)» loop is simple and improves times, there remains the problem of synchronisation times being too high, so one must think about rethinking the algorithm using a divide-and-conquer strategy and synchronising the data.
It is evident that if we paralysed «for \(i\)» and «for \(j\)» in divide-and-conquer mode, we would not be able to improve execution times as the synchronisation times are equal to or greater than those of the dummy version.

On the other hand, applying a divide-and-conquer to the outer for which then calls only certain parts of the inner for could change the number of synchronisation times.

Looking more closely at the dependencies, it is evident that:
\begin{enumerate}
    \item cell  \((h,h)\) is self-dependent.
    \item \(h\)-th row depends on itself and the previously computed  \((h,h)\) cell
    \item \(k\)-th column depends on itself and the previously computed  \((h,h)\) cell
    \item The rest of the matrix blocks as each of them depends on the \(h\)-th
    block of its row and the \(h\)-th block of its column.
\end{enumerate}

Performing the calculations in this order satisfies all the dependencies seen in \cref{fig:data-dependencies-3}, and one can start from this list to realise the parallel algorithm. \cref{fig:data-dependency-external-loop-parallel} shows the above steps.

\begin{figure}
    \centering
    \begin{subfigure}[t]{0.33\textwidth}
        \centering
        \includegraphics[width=\textwidth]{media/data_dependencies_4}
        \caption{Graphical representation of the dependencies of the algorithm to calculate the cell values for each \(h\)}
        \label{fig:data-dependencies-4}
    \end{subfigure}
    \hfill
    \begin{subfigure}[t]{0.45\textwidth}
        \centering
        \includegraphics[width=\textwidth]{media/data_dependencies_5}
        \caption{bbb}
        \label{fig:data-dependencies-5}
    \end{subfigure}
    \caption{ccc}
    \label{fig:data-dependency-external-loop-parallel}
\end{figure}

% nostra parallelizzazione
% analisi performance
% grafici carini
